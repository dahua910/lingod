\documentclass[a4paper,10pt]{article}
\usepackage{CJKutf8}
\usepackage[hmargin={2.5cm,1.5cm},vmargin={1.5cm,1.5cm},includeheadfoot]{geometry}
\usepackage{fancyhdr}
\usepackage{verbatim}
\usepackage{color}
\usepackage{framed}
\usepackage{titlesec}
\setcounter{secnumdepth}{5}
\setcounter{tocdepth}{3}
\usepackage{indentfirst}
\usepackage{parskip}
%\usepackage[unicode]{hyperref}
\usepackage[unicode,colorlinks=false,pdfborder={0 0 0}, bookmarksnumbered=true,bookmarksopen=true,bookmarksopenlevel=2]{hyperref}

\usepackage[11pt]{moresize} % 字体
\usepackage[T1]{fontenc}

\usepackage{fancybox} % 支持文本加框

\usepackage{array} % 制表环境
\usepackage{longtable} % 长表格环境

\usepackage{multirow} % 纵向合并单元格

\usepackage{graphics}
\usepackage{graphicx}
\usepackage{here} %图片位置

%opening
\title{diy自己的debian操作系统}
\author{itcook}

\begin{document}
\begin{CJK}{UTF8}{gbsn}

\maketitle

\begin{abstract}
从0开始,自己动手,制作自己的debina操作系统,让你认识操作系统组成的全过程,加深对操作系统的了解。让你熟悉开源软件的开发流程,深入体会开源的魅力,让你学会利用开源软件来解决问题。本书还介绍了多个有用的工程问题,和解决反案,也适用于其他系统。
\end{abstract}

\CJKindent
\newpage
\tableofcontents
\setcounter{tocdepth}{3}



\newpage
\begin{center}
\begin{tabular}{|c|l|l|}
    \hline
    修订号 & 时间 & 内容 \\
    \hline
    1 & 2010.11.11 & 文档初始化,纲要 \\
    \hline
    2 &  &  \\
    \hline
    3 &  &  \\
    \hline
    4 &  &  \\
    \hline
\end{tabular}
\end{center}
\newpage

\part{制作iso光盘}
\chapter{制作iso光盘}
\section{下载代码仓库}
\subsection{apt-mirror}
\subsubsection{}

\chapter{生成索引}

\section{apt-ftparchive}
\subsection{}
\subsubsection{}

\section{apt-scanpackage}
\subsection{}
\subsubsection{}

\section{apt-scansource}
\subsection{}
\subsubsection{}

\section{mini-dinstall}
\subsection{}
\subsubsection{}

\chapter{编译}
\section{单个包编译}
\subsection{}
\subsubsection{}
\section{整体编译}
\subsection{}
\subsubsection{}

\part{key验证机制}
\chapter{基本原理}
\section{概述}
\subsection{基本原理}
\subsubsection{}
\section{生成密钥}
\subsection{}
\subsubsection{}
\section{常用命令}
\subsection{}
\subsubsection{}
\section{debian密钥工具}
\subsection{}
\subsubsection{}

\part{网络安装}
\chapter{基本原理}
\section{syslinux和pxe}
\subsection{基本流程}
\subsubsection{}
\section{dhcp}
\subsection{}
\subsubsection{}
\section{源服务器}
\subsection{}
\subsubsection{}
\section{preseed.cfg}
\subsection{}
\subsubsection{}

\part{远程升级}
\chapter{}
\section{}
\subsection{}
\subsubsection{}

\part{参与debian开发}
\chapter{debian开源组织结构}
\section{}
\subsection{}
\subsubsection{}



\end{CJK}
\end{document}
